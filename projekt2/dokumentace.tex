\documentclass[a4paper, 10pt]{article}
\usepackage[left=1.5cm,text={18cm, 25cm},top=2.5cm]{geometry}
\usepackage[utf8]{inputenc}
\usepackage[czech]{babel}
\usepackage{setspace}
\usepackage{times}
\usepackage{sectsty}

\sectionfont{\fontsize{12}{15}\bfseries}
\subsectionfont{\fontsize{11}{15}\bfseries}


\begin{document}
    \noindent
    Dokumentace úlohy do předmětu IPK: Chatovací client 2016/2017 \\*
    Jméno a příjmení: Jiří Matějka \\*
    Login: xmatej52

    \section{Zadání, použité funkce a třídy}
        Úkolem bylo vytvořit program v jazyce C/C++ schopný odesílat a příjmat
        zprávy na/ze serveru.
    \section{Postup řešení}
        \subsection{Zpracování parametrů programu}
            Parametry (argumenty) programu jsou zpracovány pomocí jednoduché funkce
            a hodnoty (jméno uživatele a IP adresa serveru) jsou předány odkazem.
        \subsection{Komunikace se serverem}
            Hlavní vlákno programu se připojí na server, následně vytvoří 2 vlákna
            a uspí se. Příjmání a odesílání zpráv zajišťují právě 2 nová vlákna,
            každé se pohybuje v jiné funkci. Jedno načítá vstup z klávesnice
            a druhé vlákno příjmá zprávy a vypisuje je na standartní výstup.
        \subsection{Řešení chybových stavů}
            Pokud během programu nastane nějaká chyba, například neodeslání zprávy
            na server, program se neukončí, ale pouze vypíše varování na chybový
            výstup. Ukončování programu je řešeno pomocí odchycení signálu
            SIGKILL (ctrl + c) a aby bylo zabráněno chybám typu "memory leaks",
            jsou vlákna globální proměnou a ukončena pomocí destruktoru. Dále
            je také ve funkci odeslána poslední zpráva na server "user logged out".
        \subsection{Testování}
            Program byl zprvu testován pomocí vlastního serveru,
            následně na referenčním serveru a po dokončení projektu byl server
            spuštěn spolu s programem na referenčním virtuálním stroji.
\end{document}